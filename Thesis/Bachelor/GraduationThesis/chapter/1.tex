
本章では,本研究で扱うARマーカを含む2次元コードの概要を示す.

\section{2次元コード}

2次元コードとは,1方向のみに情報を持たない1次元バーコードに対して,水平方向と垂直方向の2方向に情報を持つ事が可能な表示方式のコードである.バーコードと比較をすると記録できる情報量が多くなり,面積当たりの情報密度が高いため,コード化するデータが同一の場合2次元コードの方が表示面積が小さくなる.その為,バーコードは識別コードとして使用される事に対して,2次元コードは大容量データ媒体として使用することが可能である.




\subsection{2次元コードの認識}
2次元コードは主にスタック型2 次元コードとマトリクス型2 次元コードの2種類に分けられる.

\begin{itemize}
\item スタック型2次元コード \\
シンボルキャラクタまたはデータコードワードと呼ばれるバーコードシンボルが情報の基本単位となっている.
行の情報を表したロウインジケータが配置されており,どの行からでも読み込める.スタート・ストップコードに囲まれている.バーコードと同様にバーの幅で情報を表すため,レーザスキャンによる読み込みも原理的に可能である.
スタック型2次元コードの代表例として,PDF417やCODE49などがある.
\end{itemize}

\begin{itemize}
\item マトリクス型2次元コード \\
セルと呼ばれる正方形や点を格子状に配列した構造を持つ2次元コードであり,一般的な外形は正方形である.
2次元コードの位置検出を行うため,正方形の枠やL字のフレームで囲われていたり,ファインダパターンと呼ばれる特徴的なマークがシンボルのなかに配置されている.セルの配置パターンを画像処理によりデコードし,
カメラまたは2次元CCD素子内蔵のリーダで読み込みを行う.
これにより,シンボルの方向に影響されることなく,全方向で読み取ることが可能になる.
マトリクス型2次元コードの代表例として,QR コード,DataMatrixや本研究で使用するARマーカなどがある.
\end{itemize}





\subsection{2次元コードの問題点}
2次元コードには汚れや傷などのデータの欠損に対して,
読み込んだデータを元のデータに復元するリードソロモン法と呼ばれる数学的エラーを検出し訂正する手法を取り入れた,誤り訂正機能がついているものが多くあることから,ある程度の汚れや傷などであれば読み取ることができる.
しかし,変形していたり,汚れや傷などによってデータの欠損が大きくあり,読み取れなかった場合に,1次元のバーコードであればヒューマンリーダブルと呼ばれる手動で入力出来る数字が付与されているが,2次元コードは情報量が多くヒューマンリーダブルを付与することができない.その為,2次元コードは読み取れなかった場合,情報を得ることができなくなってしまうため,2次元コードは正確に読み取る必要性がある.









\section{ARマーカ}
本研究で使用するARマーカについて説明する.
ARマーカには,「マーカタイプ」,「マーカレスタイプ」,「GPSタイプ」の3種類に分けられる.

\newpage


\begin{itemize}
\item マーカタイプ \\

1つ目の「マーカタイプ」とはマーカを必要とするもので,実際にARマーカを張り付けカメラに写して使うタイプである.ARマーカにはイラストや写真をマーカとして使用することもできるが,色や境界線がはっきりとしないと精度が落ちてしまうため,正方形の黒枠に囲まれた白黒の図\ref{マーカ}に示すような図形が一般的である.
黒枠でARマーカであると認識し,枠内の図柄のパターンでマーカの区別を行う.
本研究ではマーカタイプのARマーカを使用し実験を行った.
      \begin{figure}[htpp]
      \begin{center}
      \includegraphics[width=40mm]{figure/eps/ARマーカ.eps}
      \caption{ARマーカ.}
      \label{マーカ}
      \end{center}
      \end{figure}


\end{itemize}



\begin{itemize}
\item マーカレスタイプ \\



2つ目の「マーカレスタイプ」は基本的な機能はマーカタイプと同様であるがマーカを必要としないことが特徴である.物理的にマーカが配置できない場合でも情報を付加できることが魅力であるが,計算量がおおきくなってしまいハードウェアに一定の能力が必要となる.そのため専門的な知識も必要になり技術的な難易度が高い. 


\end{itemize}


\begin{itemize}
\item GPSタイプ \\


3つ目の「GPSタイプ」はGPSを用いて場所に応じて不可情報を加えるものである.GPSなどの位置情報だけでなく磁気センサや加速度センサなども併用することで情報やサービスを提供する場所やタイミングを決めている.しかし,GPSの機能やGPSを受け取る端末の性能の精度により影響を受けてしまうため,環境によっては誤差が生じてしまう可能性がある.



\end{itemize}














