現在QRコードやARマーカなどの2次元コードは,製造での工程管理,
製品ピッキング棚卸やロボット認識機能等の広い分野で利用されている.
2次元コードの特徴として,シンボルと呼ばれる特殊なパターンにより,
どの視点からでも背景模様の影響を受けない,高精度な検出が可能である.
さらに2次元コードの大きさを事前に定義することにより,張り付けられている物体の位置,
姿勢を推定することが可能である.ロボットが物体検出を行う時に2次元コード
を使用することなくパターン認識や機械学習を用いた物体検出の研究も取り組まれているが,
画像から物体の検出・姿勢推定を高精度に認識することが困難である.2次元コードを用いた
認識であれば,高速かつ高精度な検出・姿勢推定が可能である.

しかし,2次元コードを使用する前提条件として,
平面に張り付ける事が挙げられる.
この条件以外の,曲面や角に張られた2次元コードは歪みにより見え方の変化を引き起こし,
認識精度が低下する問題を抱えている.2次元コードに歪みを補正する機能はあるが,補正には限界がある.
試しに円柱などの平面でないものに2次元コードを貼り認識を試みると,歪みにより正確に認識ができない場合がある.平面状でない2次元コードの認識は困難であり,この問題を解決するための研究がされている.

そこで,本研究では,Augument Autoencoder(AAE)を用いた変形ARマーカの平面化及び姿勢推定を提案する.変形したARマーカの画像をAAEに入力し,歪みを取り除いた平面状のARマーカ画像を生成する.そして,変形ARマーカの姿勢推定を行う.


本論文の以下の構成は次のようになっている.第 1 章では,2 次元コードの種類と AR
マーカの認識について述べる.第 2 章では,曲面に貼られた変形 AR マーカを機械学習に
より推論し,推論結果を用いて正面から観測した AR マーカー画像を生成する手法を提案
する.第 3 章では,学習データの作成方法について述べる.最後に,第 4 章で評価実験に
ついて述べる.