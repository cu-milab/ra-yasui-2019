近年,QRコードやARマーカに代表される2次元コードは,
製造での工程管理やロボット認識機能の広い分野で利用されている.
2次元コードの特徴として,シンボルと呼ばれる特殊なパターンにより,
どの視点からでも検出精度に影響を受けない高精度な検出が可能である.
さらに2次元コードの大きさを事前に定義することにより,張り付けられている物体の位置,
姿勢を推定することが可能\cite{1}\cite{2}\cite{3}\cite{4}である.ロボットが物体検出を行う時に2次元コード
を使用することなくパターン認識や機械学習を用いた物体検出の研究も取り組まれているが,
画像から物体の検出・姿勢推定を高精度に認識することが困難である.2次元コードを用いた
認識であれば,高速かつ高精度な検出・姿勢推定が可能\cite{5}\cite{6}\cite{7}である.

しかし,2次元コードの使用条件として,
平面に張り付ける事が挙げられる.
この条件以外の,曲面や角に張られた2次元コードは歪みにより見え方の変化を引き起こし,
認識精度が著しく低下する問題を抱えている.2次元コードには歪みを補正する機能が存在するが,補正には限界がある.
円柱などの平面でないものに2次元コードを貼り認識を試みると,歪みにより正確に認識ができない場合がある.平面状でない2次元コードの認識は困難である.この問題は,既に機械学習により歪んだ2次元コードを検出する手法\cite{suzuki}が提案されているが,3次元姿勢の推定までは至っていない.

そこで,本研究では,機械学習を用いた変形ARマーカの位置姿勢推定法を提案する.
変形したARマーカをSSD(Single ShotMultiBox Detector)\cite{SSD}より検出し,
Augumented Autoencoder(AAE)\cite{AAE}を用いた変形ARマーカの姿勢推定を提案する.
ARマーカの写ったカラー画像をSSDに入力し,入力されたカラー画像から画像内にあるARマーカのID,座標位置を検出する.SSDの検出で得られた,ARマーカの位置情報をもとにARマーカ画像を切り取り,AAEに入力する.AAEにより歪みを取り除いた平面状のARマーカ画像への復元を行う.この時,変形ARマーカ画像の潜在変数には姿勢情報が含まれるため,データセットとして用意されている各姿勢画像の潜在変数と類似度を算出することで,変形ARマーカの姿勢推定を行うことが可能になる.


本論文の以下の構成は次のようになっている.第 1 章では,2 次元コードの種類と AR
マーカについて述べる.第 2 章では,提案手法を用いた変形ARマーカの復元及び姿勢推定について述べる.
第3章で評価実験について述べる.