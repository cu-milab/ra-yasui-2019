近年,様々な場所で QR コードに代表される 2 次元コードが利用されている.キャッシュ
レス決済などで用いられる QR コード,ゲームで用いられる AR マーカなど,我々が生活
するうえでなくてはならないものとなっている.ロボットも物体を検出するのに 2 次元
コードを用いる場合がある.検出したい物体に 2 次元コードを貼付,ロボットに搭載した
カメラで撮影した画像から 2 次元コードのを認識すること [1] で,物体の位置や姿勢を正
確に検出できる [2][3][4][5].2 次元コードを利用することなくパターン認識や機械学習に
より物体を検出する研究も取り組まれているが,画像から物体の位置や姿勢を正確に認識
することは困難であるため,実用化には多くの問題を抱えている.一方,2 次元コードを
用いた方法では,高速かつ正確に物体の位置・姿勢を認識することが可能である.

しかし,2 次元コードは平面に貼るという制約条件がある.この制約条件外で実行する
と誤認識のような問題が発生する場合がある.2 次元コードには歪みを補正する機能がつ
いているが,その機能によって歪みを補正することには限界がある.仮に円柱や球など平
面ではないものに 2 次元コードを貼付し認識を試みた場合,2 次元コードの歪みによって
正確に認識が出来ない場合がある.その為,2 次元コードは平面に貼付し,歪みが少ない
状態にしなければならない.よって,円柱や球など平面ではないものに 2 次元コードを貼
り付け,認識することは困難であり,これを解決するための研究がされている [6][7][8][9].

そこで,本研究では機械学習により変形した AR マーカを認識及び姿勢推定を行う手法を提案する.
変形したARマーカの画像をSingle Shot MultiBox Detector[10]により学習することで,歪みを持つマーカを正確に検出し,ID,座標,大きさを推定し,Augumented Autencoder[11]による学習で,歪んだARマーカの姿勢の推定を行う手法までを提案する.

本論文の以下の構成は次のようになっている.第 1 章では,2 次元コードの種類と AR
マーカの認識について述べる.第 2 章では,曲面に貼られた変形 AR マーカを機械学習に
より推論し,推論結果を用いて正面から観測した AR マーカー画像を生成する手法を提案
する.第 3 章では,学習データの作成方法について述べる.最後に,第 4 章で評価実験に
ついて述べる