\documentclass{jsarticle}
\usepackage{abst, epsf,graphicx, tabularx, ascmac}
%\usepackage{comment}

\newcommand{\bd}[1]{\mbox{\boldmath$#1$}}
\newcommand{\mysubsec}[1]{\noindent {\bf #1} \ \ }
\newcommand{\down}[1]{\raisebox{-0.0mm}{{\scriptsize #1}}}

\makeatletter
\newcommand{\figcaption}[1]{\def\@captype{figure}\caption{#1}}
\newcommand{\tblcaption}[1]{\def\@captype{table}\caption{#1}}
\makeatother


\title{機械学習を用いた変形ARマーカの位置姿勢推定}
\prof{山内 悠嗣}
\name{ER17076 安井理}

\begin{document}
\maketitle

%-------------------------------------------------------------------------
\sec{はじめに}
QRコードやARマーカに代表される2次元コードは,キャッシュレス決済や物品管理,広告,ロボットの認識等の分野において活用されている.
平面に貼り付けられた2次元コードを認識することで,高精度に3次元位置と姿勢を推定することができるが,2次元コードが変形すると認識に失敗する問題がある.
既に機械学習により2次元コードを検出する方法が提案されているが,2次元コードの3次元姿勢を推定するまでには至っていない.

そこで,本研究では機械学習による2次元コードの検出と3次元姿勢の推定方法を提案する.
提案手法は2次元コードの検出と姿勢推定の2つの処理に分けられる.本稿では2次元コードの姿勢推定について述べる.
\sec{提案手法の概要}
本研究は,SSD による変形AR マーカの検出とAAEによる3次元姿勢推定の2つの処理に分けられる.
まず,画像からSSDによりARマーカを検出する.次に,SSDにより得られたARマーカに対してAAEを適用することで変形を含まないARマーカの画像を生成する.
最後に,変形を含まないARマーカとARマーカを照合することで3次元姿勢を推定する.
なお,本稿では円柱に貼り付けることにより変形したARマーカを変形ARマーカとして扱う.

\begin{figure}[ht]
\vspace{-1zh}
\setlength{\epsfxsize}{7cm}
\centerline{\epsfbox{./eps/nagare.eps}}
\vspace{-2zh}
\caption{提案手法の概要}
\label{flow}
\vspace{-2zh}
\end{figure}

%---------------------------------------------------------------------------------------
\subsec{AAEによる変形を除去したARマーカの生成}
前段の処理により検出したARマーカの位置と大きさの情報からAugumented AutoEncoder(AAE)[1]によりARマーカの変形を除去し,エンコーダから得られる潜在変数で姿勢を推定する.

ロボットシミュレータGazeboにより事前に変形やノイズを含まないARマーカ画像(図\ref{BB}(a))と変形やノイズを加えた
ARマーカ画像(図\ref{BB}(b))を生成する.
そして,オートエンコーダ(AE)に変形やノイズを加えたARマーカ画像を入力し変形を除去したARマーカ画像(図\ref{BB}(c))を生成する.変形やノイズを含まないARマーカ画像と変形を除去したARマーカ画像の違いを吸収するようなオートエンコーダにするために,式(\ref{sonsitu})に示す損失関数$L$を最小化するように学習する.

AAEは,訓練データにノイズや変形を加えて学習し,ノイズの影響を除去した画像を生成する手法である.
AAEの流れを図\ref{BB}に示す.
提案手法では,オートエンコーダ(AE)に図\ref{BB}(b)を入力し,出力された図\ref{BB}(c)が図\ref{BB}(a)に近くなるように学習を行う.
図\ref{BB}(a)と図\ref{BB}(b)はARマーカの姿勢がペアとなるようセンサシミュレーションを用いて作成する.
AAEの学習方法は,はじめに,図\ref{BB}(b)をエンコーダーに入力し畳み込みを行い,
128次元まで圧縮された実数で表現される潜在変数$z$を得る.
次に,潜在変数をデコーダに入力し,図\ref{BB}(c)を出力する.
図\ref{BB}(a)とデコーダから出力された図\ref{BB}(c)の損失誤差を式(\ref{sonsitu})により求め,
損失誤差が小さくなるように学習を行う.
\begin{eqnarray}
\label{sonsitu}
L=\frac{1}{n}\sum_{i=1} ||x_i-x'_i||_2
\end{eqnarray}

$x$は変形やノイズを含まない画像,$x'$はオートエンコーダにより出力した画像を表す.

\begin{figure}[ht]
\vspace{-1zh}
\setlength{\epsfxsize}{7cm}
\centerline{\epsfbox{./eps/学習の流れ.eps}}
\vspace{0zh}
\caption{AAEの学習の流れ}
\label{BB}
\vspace{-2zh}
\end{figure}


\subsection{ARマーカの3次元姿勢推定}
オートエンコーダに入力した際に得られる潜在変数$z$に基づき姿勢を推定する.
事前にあらゆる姿勢のARマーカ画像をGazeboにより生成し,潜在変数群を姿勢データベース(DB)
として用意する.テスト時には,SSDにより検出したARマーカ画像をエンコーダに入力し,潜在変数を得る.
両者の潜在変数のコサイン類似度を計算し,最も類似度が高いDB中の潜在変数に対応した姿勢を
出力する.

DBには,roll[0, 360], pitch[-35, 35], yaw[-15, 15]の範囲を分解能3[$^\circ$]生成した264,124枚を用意する.


%\begin{eqnarray}
%\vspace{-2zh}
%\label{cos}
%cos_n=\frac{\vec{z}_n \vec{z}_{test}}{||\vec{z}_n|| ||\vec{z}_{test}||}
%\end{eqnarray}



%------------------------------------------------------------------------------


\sec{評価実験}


提案手法の有効性を確認するために評価実験を行う.
変形ARマーカの姿勢推定結果を平均絶対誤差(MAE)を用いて評価する.
評価を行う画像は,センサシミュレーションよりランダムな姿勢となる	
半径20, 30, 40[mm]の円柱に貼り付けたARマーカをSSDにより検出した100枚を使用する.

評価結果を表\ref{hyouka}に示す.提案手法による3次元姿勢推定精度は約2.89[$^\circ$]となった.DBの角度の分解能が3[$^\circ$]であるため,姿勢推定精度は妥当だと考えられる.また,円柱の半径が小さいほど推定精度が低下する傾向が得られた.これは円柱の半径が小さいほどARマーカの変形が大きいためだと考えられる.
%と平均絶対誤差(MAE)



\begin{table}[h]
        \vspace{0zh}
          \begin{center}
            \caption{提案手法における姿勢推定精度 MAE}
            \label{hyouka}
            \begin{tabular}{c|c|c|c|c} \hline
              円柱半径[mm]   & roll& pitch & yaw&平均 \\ \hline
              20& 4.39 & 3.06 & 2.77 &3.40\\ \hline
              30&2.76 & 2.88 & 2.58& 2.74 \\ \hline
              40&2.52 &2.65  &2.43&2.53 \\ \hline
              \end{tabular}
          \end{center}
        \vspace{-1.0zh}
\end{table}

%\begin{figure}[ht]
%\vspace{-2zh}
%\setlength{\epsfxsize}{6.5cm}
%\centerline{\epsfbox{./eps/復元.eps}}
%\vspace{-1zh}
%\caption{復元画像}
%\label
%\label{F}
%\vspace{-2zh}
%\end{figure}


%\begin{table}[h]
        %\vspace{-1.8zh}
         % \begin{center}
       %     \caption{MAE}
       %     \label{hyouka2}
     %       \begin{tabular}{l|c|c|c} \hline
   %           円柱半径[mm]   & roll& pitch & yaw \\ \hline
 %20  & 2.43° & 2.2° & 2.5° \\ \hline
         %     30  & 0° & 1° & 2.2° \\ \hline
       %       40  & 1° & 1° & 1° \\ \hline
     %         50  & 0°& 2° & 1° \\ \hline
   %           \end{tabular}
 %         \end{center}
 %       \vspace{-1.0zh}
%\end{table}





\sec{おわりに}
本研究では,変形ARマーカの検出及び姿勢推定法を提案した.その中でも,本稿では変形したARマーカの姿勢推定について述べた.本研究ではAAEに基づき,変形したARマーカにおいても高精度に3次元の姿勢を推定できることを確認した.今後は実環境における提案手法の有効性を検証する予定である.

\markboth{参考文献}{}     %% BibTeX を使う
%\bibliographystyle{jalpha}
%\newcommand{\noop}[1]{}
%\bibliographystyle{jabbrv}
%\bibliography{ref}

\begin{thebibliography}{1}
%{\scriptsize 
%\bibitem{hog+motion}
%Wei Liu,Dragomir Anguelov, Dumitru Erhan,Christian Szegedy,Scott Reed,Cheng-Yang Fu, Alexander C. Berg : ``SSD: Single Shot MultiBox Detector'', Proc. of ECCV, 2016.}

{\scriptsize 
\bibitem{pixel}
Martin Sundermeyer {\em et al.}:"Implicit 3D Orientation Learning for 6D Object Detection from RGB Images'',Proc. of ECCV, 2016.}

\end{thebibliography}
\end{document}