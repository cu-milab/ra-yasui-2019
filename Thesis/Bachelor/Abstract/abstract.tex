\documentclass{jsarticle}
\usepackage{abst, epsf,graphicx, tabularx, ascmac}
%\usepackage{comment}

\newcommand{\bd}[1]{\mbox{\boldmath$#1$}}
\newcommand{\mysubsec}[1]{\noindent {\bf #1} \ \ }
\newcommand{\down}[1]{\raisebox{-0.0mm}{{\scriptsize #1}}}

\makeatletter
\newcommand{\figcaption}[1]{\def\@captype{figure}\caption{#1}}
\newcommand{\tblcaption}[1]{\def\@captype{table}\caption{#1}}
\makeatother


\title{変形ARマーカの認識及び姿勢推定}
\prof{山内 悠嗣}
\name{ER17076 安井理}

\begin{document}
\maketitle

%-------------------------------------------------------------------------
\sec{はじめに}
現在QRコードやARマーカなどの2次元コードは,製造での工程管理,梱包,製品ピッキング棚卸,小売り業でのキャッシュレス決済,レジャーでの入場管理やロボット認識機能等の広い分野で利用されている.
2次元コードの特徴として,シンボルと呼ばれる特殊なパターンにより,360°どの視点からでも背景模様の影響を受けない,高精度な検出が可能である.また,数百から数千バイトの大容量の情報を埋め込むことができる.
さらに2次元コードの大きさを事前に定義することにより,張り付けられている物体の位置,姿勢を推定することが可能である.しかし,2次元コードを使用する前提条件として,平面に張り付ける事としており,曲面や角に張られた2次元コードは歪みにより見え方の変化を引き起こし,認識精度が低下する問題を抱えている.
\\そこで,本研究では機械学習により変形した AR マーカを認識及び姿勢推定を行う手法を提案する.変形したARマーカの画像をSingle Shot MultiBox Detector(SSD)[1]により学習することで,歪みを持つマーカを正確に検出し,ID,座標,大きさを推定し,Augumented Autencoder(AAE)[2]による学習で,歪んだARマーカの姿勢の推定を行う.


\sec{SSDとAAEを用いた姿勢推定}
 図\ref{flow}は,提案手法による変形ARマーカの認識及び姿勢推定の流れを示す.本研究ではARマーカの検出と認識に物体検出手法であるSingle Shot MultiBox Detector(SSD)を学習することによりARマーカの画内座標位置,カテゴリの種類を求める.ARマーカの姿勢をAugumented Autencoder(AAE)を学習することによって推定を行う事ができる.提案手法ではSSDとAAEにより変形ARマーカの大きさ,姿勢を推定できるように拡張する.

\begin{figure}[ht]
\vspace{-4zh}
\setlength{\epsfxsize}{7.5cm}
\centerline{\epsfbox{./eps/流れ.eps}}
\vspace{3zh}
\caption{提案手法の流れ}
\label{flow}
\vspace{-1.0zh}
\end{figure}

%---------------------------------------------------------------------------------------
\sec{AAEの構造}
ネットワークは,入力画像128$\times$128のRGB画像を5$\times$5の畳み込みstride2である,
AAEの学習の流れを図(\ref{BB})に示す.学習したい元画像となる平面状のARマーカの貼られた円柱の画像(a)を正解画像として用意し,元画像と同じ姿勢のARマーカを円柱に沿うように貼り付けた画像(b)を入力としてAutencoderにかけ出力(c)は(a)と誤差を取り(a)を再現するように学習を行う.



\begin{figure}[ht]
\vspace{-5zh}
\setlength{\epsfxsize}{7.5cm}
\centerline{\epsfbox{./eps/学習の流れ.eps}}
\vspace{5zh}
\caption{AAEの学習}
\label{BB}
\vspace{-1.0zh}
\end{figure}




%---------------------------------------------------------------------------------
\sec{AAEによる姿勢推定}
AAEは物体の三軸方向を推定するためのAutencoderである.推定までの全体的な流れを図(\ref{GG})に示す.
Autencoderにより各姿勢情報を含むターゲット画像の潜在変数を学習時に蓄積しておき,検出時に獲得されたバウンディングボックスを切り取り,Autencoderに入力し得られた潜在変数と学習時に蓄積された潜在変数との,
コサイン類似度を求めることによって最も近い値の潜在変数の姿勢情報を物体姿勢として推定する.

\begin{figure}[ht]
\vspace{-5zh}
\setlength{\epsfxsize}{7.5cm}
\centerline{\epsfbox{./eps/姿勢推定.eps}}
\vspace{5zh}
\caption{AAEによる姿勢推定}
\label{GG}
\vspace{-1.0zh}
\end{figure}

%------------------------------------------------------------------------------


\sec{評価実験}

実際の検証用画像の物体姿勢とAAEを用いた姿勢推定時との誤差を比較.姿勢推定誤差は,下記の表に示すようにマーカの種類,円柱の太さの変化対して精度の変化はなく高い精度で認識が可能である.

ar\_track\_alvar認識可能な円柱の半径とAAEの認識可能な円柱の半径を比較ではar\_track\_alvarは半径50㎝の円柱のみ認識可能であるが,AAEを用いた姿勢推定では20㎝の円柱まですべて認識可能であった.


\sec{おわりに}
本研究では,変形ARマーカの認識及び姿勢推定を提案し,機械学習によってARマーカの座標位置,姿勢を推定できる事を確認した.今後は,提案手法のよるリアルタイムでの三次元位置・姿勢の推定を研究を行う予定である.

\markboth{参考文献}{}     %% BibTeX を使う
%\bibliographystyle{jalpha}
%\newcommand{\noop}[1]{}
%\bibliographystyle{jabbrv}
%\bibliography{ref}

\begin{thebibliography}{1}
{\scriptsize 
\bibitem{hog+motion}
Wei Liu,Dragomir Anguelov, Dumitru Erhan,Christian Szegedy,Scott Reed,Cheng-Yang Fu, Alexander C. Berg : ``SSD: Single Shot MultiBox Detector'', Proc. of ECCV, 2016.}

{\scriptsize 
\bibitem{pixel}
Martin Sundermeyer {\em et al.}: ``Implicit 3D Orientation Learning for 6D Object Detection from RGB Images'', 
Proc.of ECCV, 2016.}

\end{thebibliography}
\end{document}